\documentclass{article}
\usepackage[utf8]{inputenc}
\usepackage{todonotes}
\usepackage{fullpage}

\title{Quality Management Plan}
\author{Jensenligan}
\date{February 2014}

\begin{document}

\maketitle

\section{Todo}


Copy paste these to a valid section, rewrite a bit and if not complete add a slash todo \\
http://latexforhumans.wordpress.com/2009/03/13/todonotes/).\\If you dislike the document structure change it, David
\\
Simply think, what tasks can/do we do to directly improve quality of the product. Or, what are the factors that somehow influence the quality of the product. Everything is valid!!
\begin{itemize}
\item Add more todos, increase detail of todos
\item How using pivotal improves document-ability and traceability etc.
\item How using git improves x.
\item How using scrum improves y (see scrum pm).
\item How using a (frequent) demo based development improves z (right prioritisation of functionality etc) (see scrum pm)
\item Unit tests!!!! A lot.
\end{itemize}

%\tableofcontents 

\section{Developer interest and Productivity}
We see developer interest and productivity as our main factor of influence over the quality of the device. 
The development of the device is done by a small team of 6 members, which means that all members have extremely high influence over the final product.
The performance of our team is directly relative to the productivity of each member.
It is therefore very important that each member feels connected with the product, its purpose and eager to work with it. 

Therefore we internally develop a product (the laser simulated turret) which may not be seen as either serious or interesting to the external viewer.
The laser simulated turret, which is nothing more than a toy product, is however a product that all developers feel closely engaged and motivated to work with.

However, the projects external identity (the intimidating security camera) is more formal and has a more direct customer.
This allows for all formalities regarding customer discovery and product verification to be executed, all-while keeping the developers interested and entertained at work.

Following the previous statement that developer productivity directly increases software quality, this section discusses elements prioritized to improve on this productivity.

\subsection{Development process and tools}
Please note that for classical development, more in depth verification, testing and general quality management tasks need to be executed and documented.
However, by working in scrum with a focus on agile practises, there is a much more pressure on our development process and its tool-set.
Following the agile manifest, "Our highest priority is to satisfy the customer through early and continuous delivery of valuable software.".

\subsubsection{Git}
By using git we can avoid loss of progress due to crashes and lost code.
Git also makes it easier to work on the same project without having to manually merge changes.


\section{Code quality}

\subsection{Modularity and Object oriented design}
By focusing on modularity, that is, when logically separable parts of the product is also encapsulated and separated in code.
We directly increase the maintainability and flexibility of our code.
It gets easier to overview as well as understand and edit. In other languages, encapsulating via classes is the only way to go. However, in python, only separating using multiple files is enough, and this will be our main goal. To try and separate code in as many files possible.

A good object oriented design maximises cohesion and minimizes dependency. Minimal dependency reduces costs of late bug detection.

\subsection{Coding standards}
By establishing a strict coding standard we make sure that our codebase is easily understood and easily maintained.
Our coding standard includes rules such as, name variables, classes and objects according to what they are and 


\section{Product quality}

\section{Tasks}
Tasks thar will reccur to improve quality
Estimated hours....
\section{Responsibilities}
By working in scrum, all responsibility, including quality assurance's is equally spread over team members.
It does however lie in the responsibility of our product owners that the backlog stories they add and prioritize lie in the best interest of the customer.
\section{Verification management plan}
\section{Standard practices}
\section{Faults management plan}
\section{Correction costs and early detection}
Although a good object oriented design minimizes dependency, all software will contain parts that are transitively dependant on others. This often means that if a bug is created but not discovered, there are risks that continued work of the software becomes dependant on the bug. Hence, the later a bug is discovered, the more dependency and the higher cost.

\subsection{Testing}
\subsection{Problem reporting and corrective action}
Bugreport/handling
\section{Training}
How we will improve our skills, strategies
\section{Risk management}
Adverse event
\section{}

The device function should be precise with an error margin of +-5 cm from the desired target area. 

\subsection{Safety}

\subsection{Usability and ease of use}
The user should not be required to have any experience with ssh or Linux to use the final product. (discuss, for example :) For configuration, only editing a config file found on the sd-card is required.

\end{document}
